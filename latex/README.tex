% Options for packages loaded elsewhere
\PassOptionsToPackage{unicode}{hyperref}
\PassOptionsToPackage{hyphens}{url}
%
\documentclass[
]{article}
\usepackage{amsmath,amssymb}
\usepackage{iftex}
\ifPDFTeX
  \usepackage[T1]{fontenc}
  \usepackage[utf8]{inputenc}
  \usepackage{textcomp} % provide euro and other symbols
\else % if luatex or xetex
  \usepackage{unicode-math} % this also loads fontspec
  \defaultfontfeatures{Scale=MatchLowercase}
  \defaultfontfeatures[\rmfamily]{Ligatures=TeX,Scale=1}
\fi
\usepackage{lmodern}
\ifPDFTeX\else
  % xetex/luatex font selection
\fi
% Use upquote if available, for straight quotes in verbatim environments
\IfFileExists{upquote.sty}{\usepackage{upquote}}{}
\IfFileExists{microtype.sty}{% use microtype if available
  \usepackage[]{microtype}
  \UseMicrotypeSet[protrusion]{basicmath} % disable protrusion for tt fonts
}{}
\makeatletter
\@ifundefined{KOMAClassName}{% if non-KOMA class
  \IfFileExists{parskip.sty}{%
    \usepackage{parskip}
  }{% else
    \setlength{\parindent}{0pt}
    \setlength{\parskip}{6pt plus 2pt minus 1pt}}
}{% if KOMA class
  \KOMAoptions{parskip=half}}
\makeatother
\usepackage{xcolor}
\setlength{\emergencystretch}{3em} % prevent overfull lines
\providecommand{\tightlist}{%
  \setlength{\itemsep}{0pt}\setlength{\parskip}{0pt}}
\setcounter{secnumdepth}{-\maxdimen} % remove section numbering
\usepackage{mathtools}
\usepackage{braket}
\usepackage{csquotes}
\ifLuaTeX
  \usepackage{selnolig}  % disable illegal ligatures
\fi
\IfFileExists{bookmark.sty}{\usepackage{bookmark}}{\usepackage{hyperref}}
\IfFileExists{xurl.sty}{\usepackage{xurl}}{} % add URL line breaks if available
\urlstyle{same}
\hypersetup{
  pdftitle={The Meßthaler-Wulff Project},
  pdfauthor={Julia Meßthaler},
  hidelinks,
  pdfcreator={LaTeX via pandoc}}

\title{The Meßthaler-Wulff Project}
\author{Julia Meßthaler}
\date{}

\begin{document}
\maketitle

Blazingly fast code for finding all crystals (subsets of a graph) that
can be constructed using only transformations that locally minimize
surface energy.

\hypertarget{the-problem}{%
\subsection{The Problem}\label{the-problem}}

Let \((G, E)\) be some graph and \(\eta: G \rightarrow \wp(G)\) denote
the neighbors of a given node, defined as \[
  \eta(n) \coloneq \{ n_0 \in G \mid \{n_0, n\} \in E \}
\]

Now we can define a crystal as \(c \subset G\) with \(c\) finite. This
allows us to define its complement \(\bar c \coloneq G \setminus c\) and
the set of all crystals
\(C \coloneq \{ c \subset G \mid c \text{ finite} \}\).

Now we can define the surface energy of the crystal \(c\) (or arbitrary
subsets of \(G\)) as \[
  \xi_{c} \coloneq \sum_{n \in c} f_{\bar c}(n)
\] where \(f_M(n)\) denotes the \textquote{friendliness} of the node
\(n\) or how many friends it has 4defined as \[
  f_M(n) \coloneq \# \{ n_0 \in \eta(n) \mid n_0 \in M4 \}
\]

The idea now is to find crystals \(c\) such that
\(\frac{\xi_{c}}{\# c}\) is optimal.

\hypertarget{the-crystal-graph}{%
\subsection{The Crystal Graph}\label{the-crystal-graph}}

We can impose a graph structure on \(C\) with the edges \[
  T \coloneq \{ \{ c, c \setminus \{n\} \} \mid c \in C \text{ and } n \in c \}
\] we call these the transformations and \((C,T)\) the transformation
graph.

If we want to discover optimal crystals, then we must efficiently walk
this graph. Since this graph is very dense and very large, we must first
discuss some optimizations:

\begin{itemize}
\tightlist
\item
  \protect\hyperlink{efficient-transformations}{\(O(1)\)
  transformations}
\item
  \protect\hyperlink{pruning-bad-transformations}{Pruning bad
  transformations}
\item
  \protect\hyperlink{exploiting-symmetries}{Exploiting Symmetries}
\item
  \protect\hyperlink{pruning-bad-crystals}{Pruning bad crystals}
\end{itemize}

\hypertarget{efficient-transformations}{%
\subsection{Efficient Transformations}\label{efficient-transformations}}

In the code this is achieved using the stateful class
\texttt{AdditiveSimulation} that walks along the transformation graph.
This class keeps track of two important properties, namely \(f_c(n)\)
and \(m(n)\) for (almost) all \(n \in G\), in practice \(f_c(n)\) will
default to \(0\) and \(m(n)\) to \(1\). \(m(n) \in \{0,1\}\) is called
the mode of the node \(n\) and is defined as \[
m(n) \coloneq
\begin{cases}
0 & \text{if } n \in c \\
1 & \text{if } n \in \bar c
\end{cases}
\] We call \(1\) the forwards mode and \(0\) the backwards mode, as
nodes that we might add to our current crystal will be in \(\bar c\) and
nodes we might remove are in \(c\).

We can also define the mode sign of a node as
\(s(n) \coloneq 2 \cdot m(n) - 1\).

Now let \(n\) be the node that defines our transformation, we can update
\(m(n)\) like so \(m'(n) = 1 - m(n)\). The friendliness \(f_c(n)\) does
not change. Instead, the friendliness of each neighbor must be updated.
So for each \(n_0 \in \eta(n)\): \(f_c'(n_0) = f_c(n_0) + s(n)\). We can
also define an update rule for the energy
\(\xi_c' = \xi_c + s(n) \cdot f_{\bar c}(n) - s(n) \cdot f_{c}(n) \text{ todo}\).
\textbf{TODO: Explanation}

\hypertarget{pruning-bad-transformations}{%
\subsection{Pruning Bad
Transformations}\label{pruning-bad-transformations}}

Most transformations are pretty bad, for example in a solid crystal,
removing a node in the center will only increase the energy, to combat
this we will only consider locally optimal transformations. For this we
define
\(\delta(n) \coloneq \xi_{c'} - \xi_{c} = s(n) \cdot f_{\bar c}(n) \text{ todo}\),
the energy difference for this transformation.

Now we look for nodes \(n\) such that no nodes \(n_0\) exist with
\(m(n_0) = m(n)\) and \(\delta(n_0) < \delta(n)\). These are the locally
optimal transformations.

This is achieved in \(O(1)\) using the \texttt{PriorityStack} class, a
priority queue optimized for this specific use case. We have two
instances for our graph walking \texttt{AdditiveSimulation}, one for
backwards mode and one for forwards mode. Depending on which mode is
then queried, we query the appropriate instance for all nodes that have
minimal \(\delta(n)\).

\hypertarget{exploiting-symmetries}{%
\subsection{Exploiting Symmetries}\label{exploiting-symmetries}}

Let \(H\) be a group action on \(G\). This is only useful to us if the
action commutes with the neighborhood function as in
\(\eta(h(n)) = h(\eta(n))\) for all \(h \in H\). This not only means
that the energy is also invariant under \(H\), but also the possible
next locally optimal transformations, meaning we can run our simulation
only on canonical representatives of each equivalence class of \(G/H\).

We call a function \(\chi_H: G \to H\) a characteristic for the group
action \(H\) if for all \(h \in H\) and \(g \in G\)
\[\chi_H(h(g)) = h \circ \chi_H(g)\]

\hypertarget{pruning-bad-crystals}{%
\subsection{Pruning Bad Crystals}\label{pruning-bad-crystals}}

\begin{center}\rule{0.5\linewidth}{0.5pt}\end{center}

\end{document}
